% !TEX TS-program = xelatex
% The "real" document content comes below...
% !TEX TS-program = xelatex
% !TEX encoding = UTF-8 Unicode
% \ProvidesClass{classname}[2017/02/27 notes]
\documentclass[10pt]{article}%__________________________________________________________________Page geometry
% \usepackage[letterpaper,margin=0.6in,marginratio=3:5,twoside]{geometry}
\usepackage[a4paper,margin=15mm,marginratio=3:5,twoside]{geometry}
% \usepackage[a5paper,margin=13mm]{geometry}
%\usepackage[letterpaper,margin=0.6in,marginratio=3:5,twoside]{geometry}
% \usepackage[paperwidth=5.5in, paperheight=8.5in, margin=13mm]{geometry}
% \usepackage{relsize}
% \relscale{0.3} % or whatever scaling is desired
%\geometry{landscape} % sets the page to landscape orientation

%\usepackage{ulem}
% \usepackage{fancyhdr} % This should be set AFTER setting up the page geometry
%\pagestyle{fancyplain} % options: empty , plain , fancy
%_____________________________________________________________________MATH STUFF
\usepackage{amsmath,amssymb,amsopn}% read amsldoc.pdf for more details
\usepackage{fontspec}
\usepackage{unicode-math}
%\DeclareMathOperator{\}{}
\DeclareMathOperator{\arccot}{arccot}
\DeclareMathOperator{\arcsec}{arcsec}
\DeclareMathOperator{\arccsc}{arccsc}
\DeclareMathOperator{\sech}{sech}
\DeclareMathOperator{\csch}{csch}
\DeclareMathOperator{\arcsinh}{arcsinh}
\DeclareMathOperator{\arccosh}{arccosh}
\DeclareMathOperator{\arctanh}{arctanh}
\DeclareMathOperator{\arccoth}{arccoth}
\DeclareMathOperator{\arcsech}{arcsech}
\DeclareMathOperator{\arccsch}{arccsch}
%\DeclareMathOperator{\min}{min}
%\DeclareMathOperator{\max}{max}
%____macros to make quick note taking easier all escapes under 3 chars
\def\t{\text }
\def\parti{\partial}
\def\o{\over}
\def\({\left( }
\def\){\right) }
\def\[{\left[ }
\def\]{\right] }
% \def\l{\left. }
% \def\r{\right. }
%\usepackage{leqno}
%__________________________________________________________________Font settings
% \usepackage[scaled=.92]{helvet}
% \usepackage{times}
% \usepackage{morefloats}
\usepackage{xcolor}
\usepackage{titlesec}
% \defaultfontfeatures{Ligatures=TeX}
\setsansfont{NotoSans-Light.ttf} % Set sans serif font
% \setsansfont{DejaVu Sans, ExtraLight} % Set sans serif font
% \setsansfont{DejaVu Sans ExtraLight}
% [    Extension = .ttf,
%    UprightFont = *,
%       BoldFont = *-Bold,
%     ItalicFont = *-Italic,
% BoldItalicFont = *-BoldItalic,
% ]
% \setmainfont{Latin Modern Math} % Set serifed font
% \setmainfont{DejaVu Math TeX Gyre} % Set serifed font
\setmainfont{XITS} %#/usr/share/texlive/texmf-dist/fonts/opentype
[    Extension = .otf,
   UprightFont = *-Regular,
      BoldFont = *-Bold,
    ItalicFont = *-Italic,
BoldItalicFont = *-BoldItalic,
]
\setmathfont{XITSMath-Regular}
[    Extension = .otf,
      BoldFont = XITSMath-Bold,
]
% \setmathfont{Latin Modern Math} % Set serifed font

% Define light and dark Microsoft blue colours
\definecolor{MSBlue}{rgb}{.204,.353,.541}
\definecolor{MSLightBlue}{rgb}{.31,.506,.741}
\definecolor{PitchBlack}{rgb}{.0,.0,.0}

% Define a new fontfamily for the subsubsection font
% Don't use \fontspec directly to change the font
% \newfontfamily\subsubsectionfont[Color=MSLightBlue]{DejaVu Math TeX Gyre}
% Set formats for each heading level

\titleformat*{\section}{\Large\bfseries\sffamily\color{PitchBlack}}
\titleformat*{\subsection}{\Large\bfseries\sffamily\color{PitchBlack}}
\titleformat*{\subsubsection}{\Large\bfseries\sffamily\color{PitchBlack}}
% \titleformat*{\subsection}{\large\bfseries\sffamily\color{MSLightBlue}}
% \titleformat*{\subsubsection}{\itshape\subsubsectionfont}

%______________________________Multicol http://tex.stackexchange.com/a/3987/1244
\usepackage{multicol}
\usepackage{ifthen}
% \let\oldsection\section
% \renewcommand*{\section}[1]{%
%     \ifthenelse{\equal{\@currenvir}{multicols}}{\end{multicols}}{}
%     \ifthenelse{\thesection = 0}{}{\end{multicols}}%
%     \begin{multicols}{2}[\oldsection{#1}]}
% \let\oldsubsection\subsection
% \renewcommand*{\subsection}[1]{%
%     \ifthenelse{\thesubsection = 0}{}{\end{multicols}}%
%     \begin{multicols}{2}[\oldsubsection{#1}]}
% \let\oldend\enddocument
% \renewcommand*{\enddocument}{\end{multicols}\oldend}


\usepackage{pst-node}% http://ctan.org/pkg/pst-node
\usepackage{tabu}
\usepackage{natbib}
\usepackage{graphicx}
%\usepackage{hyperref}
\usepackage[colorlinks=true,linkcolor=black,anchorcolor=black,citecolor=black,filecolor=black,menucolor=black,runcolor=black,urlcolor=black]{hyperref}%\usepackage{textcomp}
%\title{Pfund Mass $= \sqrt{c^{88} 10\over G}= M_F$ }
%\title{Grand Unified Theory of Nonesense}
%\title{Physics 101}
\title{Title TBD}
\author{Roy Pfund}
%\date{} % Activate to display a given date or no date (if empty), otherwise the current date is printed
\begin{document}
\bibliographystyle{plainnat}
%\maketitle
%\begin{multicols}{2}
\maketitle

%% Abstract section.
%\begin{abstract}
%testing
%\end{abstract}
%\end{multicols}
%-------------Abstract--------------
%\begin{multicols}{2}
\begin{multicols}{2}
 in addition to $\hbar$, $G$, and $C$ those are quite interesting \& in sections 2+, Then looking into how the Stoney-Units baked out the need for Coulumb's constant $k_e$, and the elemntary charge cancels itself out.

%Ampère's force law for the magnetic force $F = {\mu _{0} I_1 I_2 l \over 2 \pi r}$ between 2 straight parallel conductors where $l$ is the length of the shorter wire, $r$ is the distance between the 2 wires, and $I_1$, $I_2$ are the direct currents carried by the wires. 

%Vacuum permittivity-the ability of a body to store an electrical charge per unit distance
%$$\text{Vacuum permittivity}={\epsilon}_0= 8.8541878128×10^{-12} {\dfrac {{\text{s}}^{2}{\cdot }{\text{C}}^{2}}{{\text{m}}^{3}{\cdot }{\text{kg}}}}$$




%\bibliographystyle{te}
%This command tells BibTeX to use the bibliography style file te.bst.  This file should be in a directory where LaTeX and BibTeX can find it.  For example, if you're using MiKTeX on Windows, then the available bst files are in a directory named something like \Program Files\MiKTeX 2.9\bibtex\bst.  If you have a bst file that is not available there, put it in a subdirectory of \<your local TeX directory>\bibtex\bst, where <your local TeX directory> might sensibly be localtexmf.
%\bibliography{research}
\subsection*{Stoney \& Planck Units}
This wasn't the 1st time units were constructed from constants; in 1881 George Johnstone Stoney published units very similar to the ones below\citep{Stoney1883}.
%Stoney Units
%length log(1.3807×10^−36)/log(2) = −119.12401
%mass log(1.859×10^−9)/log(2)= −29.0028
%Time log(4.605×10^−45)/log(2)= −147.28
%charge log(1.602×10^−19)/log(2)=−62.43
\begin{align*}
Q_S &= q_e = 1.602176634×10^{-19} C =\text{Stoney Electric charge (Q)} \\
M_{{\text{S}}}&= \sqrt{k_e ~e^2\over G} = 1.85921×10^9 kg = \text{Stoney Mass (M)} \\
L_S&=\sqrt{G~k_e~e^2\over c^4} =1.38068×10^{-36} m = \text{Stoney Length (L)} \\
T_{{\text{S}}}&=\sqrt{G~k_e ~e^2\over c^{6}} = 4.60544×10^{-45} s = \text{Stoney Time (T)}
\end{align*}
There and back again . . . if we do the same operations we would do to the ElectronVolt to get mass, length, or time; to the $E_s$ we get the original stoney units back, only this time we do occasionally need to use a dimensionless constant.
$$\epsilon _0 ={t_p^2 ~q_e^2 \over m_p~ l_p^3 ~4~\pi~ \alpha}={q_e^2 \over \hbar ~ c ~4~\pi~ \alpha};\qquad
%1/bc=(a^(-1/2)b^(1/2)c^(1/2))^(-1)(a^(1/2)b^(1/2)c^(-3/2))^(-3)(a^(1/2)b^(1/2)c^(-5/2))^(2)
k_e={1 \over 4~ \pi ~ \epsilon _ 0}={\hbar ~ \alpha ~ c \over q_e^2}$$
\begin{align*}
Q_S =Q_e \qquad\qquad E_S&= {\sqrt{\hbar ~ \alpha ~c\over G}~c^2}\\
M_{{\text{S}}} =\sqrt{{\hbar ~ \alpha ~ c \over q_e^2} ~e^2\over G}& = {{\sqrt{\hbar ~ \alpha ~c\over G}~c^2} \over c^2}= \sqrt{\hbar ~ \alpha ~c \over G}\\
L_S=\sqrt{G~{\hbar ~ \alpha ~ c \over q_e^2}~e^2\over c^4}= \sqrt{G~\hbar ~ \alpha \over c^3} &={\hbar ~ \alpha ~ c\over {\sqrt{\hbar ~ \alpha ~c\over G}~c^2}}={G\over {c^2}}\sqrt{\hbar ~ \alpha ~c\over G}\\
T_{{\text{S}}}=\sqrt{G~{\hbar ~ \alpha ~ c \over q_e^2} ~e^2\over c^{6}}= \sqrt{G~\hbar ~ \alpha \over c^5} &={\hbar  \alpha \over {\sqrt{\hbar ~ \alpha ~c\over G}~c^2}}= {G\over {c^3}}\sqrt{\hbar ~ \alpha ~c\over G}
\end{align*}
%The Stoney units $K_b$ in Stoney units = $2~\pi ~\alpha $
If we remove the dimensionless constant $\alpha$ we get the units Max Planck proposed in 1899, by combining powers of fundamental constants $\hbar$, $G$, and $c$.
\begin{align*}
M^{\beta-\alpha} L^{3\alpha+2\beta+\gamma} T^{-2\alpha-\beta-\gamma}&=\(M^{-1} L^3 T^{-2}\)^{\alpha}\(ML^2 T^{-1}\)^{\beta}\(LT^{-1}\)^{\gamma}\\
M^1 L^0 T^0=G^{-1/2}\hbar^{1/2}c^{1/2}&=m_p=\sqrt{\frac{\hbar c}{G}}\\
M^{0} L^1 T^{0}=G^{1/2}\hbar^{1/2}c^{-3/2}&=l_p=\sqrt{\frac{\hbar G}{c^3}}={G\over {c^2}}\sqrt{\frac{\hbar c}{G}}\\
M^0 L^0 T^1=G^{1/2}\hbar^{1/2}c^{-5/2}&=t_p=\sqrt{\frac{\hbar G}{C^5}}={G\over {c^3}}\sqrt{\frac{\hbar c}{G}}
%kg^{\beta-\alpha} m^{3\alpha+2\beta+\gamma} s^{-2\alpha-\beta-\gamma}&\longleftrightarrow\(kg^{-1} m^3 s^{-2}\)^{\alpha}\(kgm^2 s^{-1}\)^{\beta}\(ms^{-1}\)^{\gamma}\\
%m_p=\sqrt{\frac{\hbar c}{G}}=kg^1 m^0 s^0&=\(kg^{-1} m^3 s^{-2}\)^{-1/2}\(kgm^2 s^{-1}\)^{1/2}\(ms^{-1}\)^{1/2}\\
%l_p=\sqrt{\frac{\hbar G}{c^3}}=kg^{0} m^1 s^{0}&=\(kg^{-1} m^3 s^{-2}\)^{1/2}\(kgm^2 s^{-1}\)^{1/2}\(ms^{-1}\)^{-3/2}\\
%t_p=\sqrt{\frac{\hbar G}{C^5}}=kg^0 m^0 s^1&=\(kg^{-1} m^3 s^{-2}\)^{1/2}\(kgm^2 s^{-1}\)^{1/2}\(ms^{-1}\)^{-5/2}\\
%m_p=\sqrt{\frac{\hbar c}{G}}=M^1 L^0 T^0&=\(M^{-1} L^3 T^{-2}\)^{-1/2}\(ML^2 T^{-1}\)^{1/2}\(LT^{-1}\)^{1/2}\\
%l_p=\sqrt{\frac{\hbar G}{c^3}}=M^{0} L^1 T^{0}&=\(M^{-1} L^3 T^{-2}\)^{1/2}\(ML^2 T^{-1}\)^{1/2}\(LT^{-1}\)^{-3/2}\\
%t_p=\sqrt{\frac{\hbar G}{C^5}}=M^0 L^0 T^1&=\(M^{-1} L^3 T^{-2}\)^{1/2}\(ML^2 T^{-1}\)^{1/2}\(LT^{-1}\)^{-5/2}
 \end{align*}
 
%\begin{align*}%constants=1
%\frac{l_p}{t_p}=\frac{\sqrt{\frac{hG}{c^3}}}{\sqrt{\frac{hG}{C^5}}}= {\hbar \over l_p m_p} =c\qquad\qquad\frac{l_p^3}{{t_p}^2 m_p}=\frac{{\sqrt{\frac{hG}{c^3}}}^3}{{\sqrt{\frac{hG}{C^5}}}^2 \sqrt{\frac{hc}{G}} }=G&\\
%{m_p~l_p^2 \over t_p}=\frac{{\sqrt{\frac{hG}{c^3}}}^2 \sqrt{\frac{hc}{G}}}{{\sqrt{\frac{hG}{C^5}}}  }=\hbar&
%%= {{2.176434*10^{-8}kg \({1.616255*10^{-35}m}\)^2 }\over{5.391246E-44s}}%
%%(((2.176434*10^-8)^(1))*((1.616255*10^-35)^(2))*((5.391246*10^-44)^(-1))*((1.602177*10^-19)^(0)))=hbar
%%(((2.176434*10^-8)^(1))*((1.616255*10^-35)^(2))*((5.391246*10^-44)^(-1))*((1.602177*10^-19)^(0))×2pi)=h
%\end{align*}

%$$\epsilon _0 ={t_p^2 ~q_e^2 \over m_p~ l_p^3 ~4~\pi~ \alpha}={q_e^2 \over \hbar ~ c ~4~\pi~ \alpha}$$
%$$Q_P = \sqrt{\epsilon _0 ~ 4\pi ~ \hbar ~c}= \sqrt{{q_e^2 ~ 4\pi ~ \hbar ~c \over \alpha ~ 4~\pi~ \hbar ~ c}}= {q_e \over \sqrt{\alpha}}$$
%\begin{align*}
%\frac{l_p^2 m_p}{{t_p} }&=\frac{{\sqrt{\frac{hG}{c^3}}}^2 \sqrt{\frac{hc}{G}}}{{\sqrt{\frac{hG}{C^5}}}  }= {{\({1.616255*10^{-35}m}\)^2 2.176434*10^{-8}kg}\over{5.391246E-44s}}=\hbar\\
%\frac{l_p^3}{{t_p}^2 m_p}&=\frac{{\sqrt{\frac{hG}{c^3}}}^3}{{\sqrt{\frac{hG}{C^5}}}^2 \sqrt{\frac{hc}{G}} }=G=6.674 30 x 10^{-11}kg^{-1} m^3 s^{-2}\\
%\frac{l_p}{t_p}&=\frac{\sqrt{\frac{hG}{c^3}}}{\sqrt{\frac{hG}{C^5}}}= {\hbar \over l_p m_p} =c=299792458 ms^{-1}\\
%\end{align*}

%((6.62607015*(10^-34))/(2pi)) (299792458) (6.67430 * (10^-11))
%((((hc)/(G))^0.5)^(1))
%((((hG)/(c^3))^0.5)^(1))
%((((hG)/(C^5))^0.5)^(1))
%2.1764343427179E-8kg=((((((6.62607015*(10^-34))/(2pi))(299792458))/((6.67430 * (10^-11))))^0.5)^(1))
%1.6162550244237E-35m=((((((6.62607015*(10^-34))/(2pi))(6.67430 * (10^-11)))/((299792458)^3))^0.5)^(1))
%5.3912464483136E-44s=((((((6.62607015*(10^-34))/(2pi))(6.67430 * (10^-11)))/((299792458)^5))^0.5)^(1))
%c=((((hc)/(G))^0.5)^(0 ))((((hG)/(c^3))^0.5)^(1))((((hG)/(C^5))^0.5)^(-1))
%G=((((hc)/(G))^0.5)^(-1))((((hG)/(c^3))^0.5)^(3))((((hG)/(C^5))^0.5)^(-2))
%h=((((hc)/(G))^0.5)^(1 ))((((hG)/(c^3))^0.5)^(2))((((hG)/(C^5))^0.5)^(-1))
%c=(((1.616255*10^-35)^(1))*((2.176434*10^-8)^(0))*((5.391246*10^-44)^(-1)))
%G=(((1.616255*10^-35)^(3))*((2.176434*10^-8)^(-1))*((5.391246*10^-44)^(-2)))
%h=(((1.616255*10^-35)^(2))*((2.176434*10^-8)^(1))*((5.391246*10^-44)^(-1)))

%\columnbreak
%\columnbreak%%
%(((2.176434*10^-8)^(−1))*((1.616255*10^-35)^(3))*((5.391246*10^-44)^(−2))*((1.602177*10^-19)^(0)))
\subsection*{Speed of Sound }
``One reason for this is the strikingly
simple relationship between the limiting low-pressure speed
of sound $c_0$ in a monatomic gas and the root-mean-squared
speed of the molecules, $v_{RMS}: c_0 = \sqrt{5/9} v_{RMS}$ . 
In terms of macroscopically measurable parameters this becomes 
\begin{align*}
&c_0 =\sqrt{γ_0  N_A k_B T \over M}
&k_B = {Mc^2_0 \over γ_0 T N_A}
\end{align*}
Since $γ_0 = 5/3$ exactly for monatomic gases, and $N_A$ is known
with a relative standard uncertainty $u_R = 0.044 × 10^{-6}$ experimentally, the challenge is to measure $M$, $T$ and $c_ 0^2$ with
low uncertainty.'' \citep{Podesta_2013}

Where $T$ is the thermodynamic temperature, $p$ pressure, $V$ volume,
% absolute temperature $T$.
$γ_0$ is the ratio of the principal heat capacities of the gas in the limit of low pressure,
$k_B$ is the Boltzmann constant,
$M$ is the molar mass of the gas, 
$N$ number of molecules of gas.
The ideal gas law states that ${N T \over p V k_B}=1$
%$${N~T \over p~V }={h\over c ~ q_e}=k_b=1.380649×10^{−23} kgm^2 s^{-2}⋅K^{-1}$$
$$T_p= {l_p~electronCharge \over  t_p^{2}~2\pi }={\sqrt { {\hbar c^{5}\over G}} \over k_{\text{B}}}=1.416784(16)×10^{32}K$$

%k_b=(((((1.616255*10^-35)^(1))*((2.176434*10^-8)^(1))* )2pi)/((1.60217×10^−19)^2))
%k_b=(((((1.616255*10^-35)^(1))*((2.176434*10^-8)^(1))* )2pi)/((1.60217×10^−19)^1))
%(((2.176434*10^-8)^(0))*(((1.616255*10^-35)^(1))*((5.391246*10^-44)^(-2)))×(1.60217 × 10^(-19)))/(2pi)
%((2.176434*10^-8)^(1))*(((1.616255*10^-35)^(2))*((5.391246*10^-44)^(-2)))planck energy
%T_p&= {m_p^{0}~ l_p^{1}~ t_p^{-2}~electronCharge  \over 2\pi }={\sqrt {\frac {\hbar c^{5}}{Gk_{\text{B}}^{2}}}}=1.416784(16)×10^{32}K
%(((2.176434*10^-8)^(0))*(((1.616255*10^-35)^(1))*((5.391246*10^-44)^(−2)))×((1.60217 × 10^(-19))^(1))*((2pi)^(−1)))
%(((2.176434*10^-8)^(0))*((1.616255*10^-35)^(0))*((5.391246*10^-44)^(0))*((1.602177*10^-19)^(0)))*((2pi)^(0)))
%$$k_b={2\pi \sqrt{{\hbar c \over G}} \sqrt{{\hbar G\over c^3}} \over electronCharge}={2~\pi~ \hbar  \over c * electronCharge}$$
%(((2.176434*10^-8)^(0))*((1.616255*10^-35)^(0))*((5.391246*10^-44)^(0))*((1.602177*10^-19)^(0)))
%(((2.176434*10^-8)^(0))*(((1.616255*10^-35)^(1))*((5.391246*10^-44)^(−2)))×((1.60217 × 10^(-19))^(1))*((2pi)^(−1)))
%((((2.176434*10^-8)^(1))*(((1.616255*10^-35)^(-1))*((5.391246*10^-44)^(-2)))×(1.60217 × 10^(-19)))*((2pi)^(0)))
%((((2.176434*10^-8)^(0))*(((1.616255*10^-35)^(3))*((5.391246*10^-44)^(0)))×(1.60217 × 10^(-19)))*((2pi)^(0)))
%(((2.176434*10^-8)^(1))*(((1.616255*10^-35)^(1))*((5.391246*10^-44)^(0)))×((1.60217 × 10^(-19))^(−1))*((2pi)^(1)))
%\begin{align*}
%&{(N)({l_p~electronCharge \over  t_p^{2}~2\pi })\over (m_pl_p^{-1}t_p^{-2})(l_p^3)}={N T \over p V }=k_b={h\over c ~ q_e}={2\pi l_p m_p \over electronCharge}\\
%&k_b={2\pi(1.616255*10^{-35}m)(2.176434*10^{-8}kg)\over1.60217×10^{−19}coulombs}\\
%&{N T \over p V }={(N)(T)\over (kgm^{-1}s^{-2})(m^3)}
%\end{align*}
%\begin{align*}


%\end{align*}

\end{multicols}
\end{document}
